% Chapter Template

\chapter{Conclusiones} % Main chapter title

\label{Chapter5} % Change X to a consecutive number; for referencing this chapter elsewhere, use \ref{ChapterX}


%----------------------------------------------------------------------------------------

%----------------------------------------------------------------------------------------
%	SECTION 1
%----------------------------------------------------------------------------------------

\section{Resultados obtenidos}

Se desarrolló e implementó con éxito un sistema de monitoreo de personas aplicando conocimientos adquiridos a lo largo de todo el año de la Especialización de Inteligencia Artificial. Se alcanzó a cumplir el objetivo propuesto, el cual consistía en estudiar los movimientos que realiza una persona al ingresar y transitar un espacio durante al menos el 80\% del tiempo que la persona permanece en el espacio.

A continuación se listan los logros destacados del trabajo final:
\begin{itemize}
\item Cumplir con la planificación original.
\item Alcanzar las métricas pautadas para el sistema de seguimiento.
\item Entrar un modelo propio basado en OsNet.
\item Combinar técnicas de aprendizaje profundo con técnicas de aprendizaje automático para re-identificación de personas.
\item Re-identificar a las personas que salen de cámara o son ocluidas durante un tiempo considerable.
\item Generar material en video simulado para ensayar el sistema.
\end{itemize}


A continuación se resaltan aquellas materias de mayor relevancia para este trabajo:
\begin{itemize}
\item Gestión de Proyectos: la elaboración de un plan de proyecto para organizar el trabajo final, facilitó la realización del mismo y evitó demoras innecesarias.
\item Análisis de datos: el análisis de los datos de entrenamiento y su preprocesamiento permitieron mejorar los resultados de entrenamiento.
\item Aprendizaje automático: el uso de técnicas de segmentación y clasificación para la re-identificación de personas.
\item Visión por computadora I y II: la experiencia adquirida en el uso de algoritmos y modelos de detección de objetos fueron vitales para la realización de este trabajo.
\end{itemize}

%----------------------------------------------------------------------------------------
%	SECTION 2
%----------------------------------------------------------------------------------------
\section{Trabajo futuro}

Utilizando la experiencia adquirida en la realización de este trabajo se encontraron diferentes aspectos de mejora del prototipo, necesarios para que se convierta en un producto comerciable:

\begin{itemize}
\item Mejorar el modelo entrenado de OsNet utilizando \textit{triple-loss} y datos de diferentes orígenes.
\item Evaluar la utilización de \textit{Tensorflow Real Time} para mejorar los tiempos de cómputo de los modelos de IA.
\item Evaluar que tipo de dispositivo podría ejecutar el sistema completo en tiempo real.
\item Evaluar el uso de cámaras de mayor ángulo visual para capturar más espacio del recinto.
\item Evaluar incorporar modelos de detección de pose a fin de obtener más información de la actividad que desarrollada la persona en el recinto.
\end{itemize}

Durante la utilización de la interfaz del sistema también se encontraron aspectos a mejorar en cuanto a la usabilidad:
\begin{itemize}
\item Crear una base de datos para almacenar las métricas obtenidas en cada ensayo o ejecución.
\item Disponer la posibilidad de visualizar varias cámaras o recintos desde la misma interfaz.
\item Desarrollar una interfaz reducida para su visualización en dispositivos móviles.
\item Elaborar alertas programables relativas a que está sucediendo en el local.
\end{itemize}
