\chapter{Introducción específica} % Main chapter title

\label{Chapter2}

En este capítulo

%----------------------------------------------------------------------------------------
%	SECTION 1
%----------------------------------------------------------------------------------------

\section{Requerimientos}
\label{sec:requerimientos}

Es esta sección

\subsection{Tablas}

Para las tablas utilizar el mismo formato que para las figuras, sólo que el epígrafe se debe colocar arriba de la tabla, como se ilustra en la tabla \ref{tab:peces}. Observar que sólo algunas filas van con líneas visibles y notar el uso de las negritas para los encabezados.  La referencia se logra utilizando el comando \verb|\ref{<label>}| donde label debe estar definida dentro del entorno de la tabla.

\begin{table}[h]
	\centering
	\caption[caption corto]{caption largo más descriptivo}
	\begin{tabular}{l c c}    
		\toprule
		\textbf{Especie}     & \textbf{Tamaño} & \textbf{Valor}\\
		\midrule
		Amphiprion Ocellaris & 10 cm           & \$ 6.000 \\		
		Hepatus Blue Tang    & 15 cm           & \$ 7.000 \\
		Zebrasoma Xanthurus  & 12 cm           & \$ 6.800 \\
		\bottomrule
		\hline
	\end{tabular}
	\label{tab:peces}
\end{table}

%----------------------------------------------------------------------------------------
%	SECTION 2
%----------------------------------------------------------------------------------------

\section{Modelos de inteligencia artificial utilizados}
\label{sec:requerimientos}

\subsection{Detector Yolo}

\textit{Yolo (You Only Look once)}\footnote{\url{https://en.wikipedia.org/wiki/Raster_graphics}} es un detector...


\begin{figure}[ht]
	\centering
	\includegraphics[scale=.50]{./Figures/questionMark.png}
	\caption{Esquema de funcionamiento de Yolo.}
	\label{fig:diagramaYolo}
\end{figure}



\subsection{Seguidor Deepsort}

\subsection{Extractor de características Osnet}

%----------------------------------------------------------------------------------------
%	SECTION 3
%----------------------------------------------------------------------------------------

\section{Desafíos en el seguimiento de personas}
\label{sec:desafiosSeguimiento}

%----------------------------------------------------------------------------------------
%	SECTION 4
%----------------------------------------------------------------------------------------

\section{Zonas de interés}
\label{sec:zonasInteres}
